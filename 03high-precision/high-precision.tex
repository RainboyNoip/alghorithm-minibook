\section{高精度算法}

\subsection{ 朴素高精度加}

\begin{lstlisting}
/*============================================================================
* Title : 朴素高精度加 ver 0.0.2 -修复一个小错误
* Author: Rainboy
* Time  : 2016-04-13 20:48
* update: 2016-06-05 11:02
* © Copyright 2016 Rainboy. All Rights Reserved.
*=============================================================================*/

/* 算法思想:
 *          用字符串来存数字,模拟运算过程,字符串最长255
 *  
 *  原理:
 *      123
 *   +  456
 *   -------
 *      579
 *
 * */
#include <cstdio>
#include <cstdlib>
#include <cstring>

char sa[1000],sb[1000];
int a[1000],b[1000],c[1000];
int la,lb,lc;

int main(){
    int i;

    scanf("%s%s",sa,sb);
    la =strlen(sa);
    lb =strlen(sb);

    memset(a,0,sizeof(a));
    memset(b,0,sizeof(b));
    memset(c,0,sizeof(c));

    for(i=0;i<la;i++)   a[la-1-i] = sa[i] -'0'; //倒序存储,个位在第0个位置
    for(i=0;i<lb;i++)   b[lb-1-i] = sb[i] -'0';

    lc = la>lb?la:lb;

    for(i=0;i<lc;i++){
        c[i+1] = (a[i]+b[i]+c[i]) /10; //先算c[i+1]
        c[i] = (a[i]+b[i]+c[i]) % 10;
        /*  也可以这样写
         *  c[i] = a[i]+b[i] +c[i];
         *  if(c[i] >=10){
         *      c[i+1] = 1;
         *      c[i]-=10;
         *  }
         * */
    }

    if(c[lc] > 0) lc++; //判断有没有进位
    for(i=lc-1;i>=0;i--)
        printf("%d",c[i]);
    return 0;
}
\end{lstlisting}


\subsection{亿/万进制高精度加法}

\begin{lstlisting}
/*============================================================================
* Title : 高精度-亿进制
* Author: Rainboy
* Time  : 2016-05-09 09:17
* update: 2016-05-09 09:17
* ? Copyright 2016 Rainboy. All Rights Reserved.
*=============================================================================*/


#include <cstdio>
#include <cstring>

char sa[2500],sb[2500];

int a[1000],b[1000],c[1000];

int la,lb,lc;

const int base  = 8; //8个0,亿进制
const int base2 = 100000000;

/* 返加 位数 */
int str2num(char x[],int y[]){
    int i,j=0,k=1;
    int lenx= strlen(x);
    for(i=lenx-1;i>=0;i--){
        if(k==base2) j++,k=1;
        y[j] += k*(x[i]-'0');
        k*=10;
    }
    return j+1;
}

int main(){
    int i;
    scanf("%s%s",sa,sb);
    memset(a,0,sizeof(a));
    memset(b,0,sizeof(b));
    memset(c,0,sizeof(c));

    /* 转换数字 */
    int lena = str2num(sa,a);
    int lenb = str2num(sb,b);

    /* 开始运算 */
    int lc = lena >lenb ?lena:lenb;//max(lx,ly)

    for(i=0;i<lc;i++){ // 这里是一个需要记忆的地方
        c[i+1] =(a[i]+b[i]+c[i])/(base2);
        c[i] = (a[i]+b[i]+c[i]) % base2;
    }

    if(c[lc] >0)
        lc++;

    //输出
    printf("%d",c[lc-1]);
    for(i=lc-2;i>=0;i--)
        printf("%08d",c[i]);

    return 0;
}
\end{lstlisting}


\subsection{朴素高精度减法}

\begin{lstlisting}
/*============================================================================
* Title : 朴素高精度减法
* Author: Rainboy
* Time  : 2016-04-14 11:06
* update: 2016-06-12 09:30
* ? Copyright 2016 Rainboy. All Rights Reserved.
*=============================================================================*/

/*  原理模拟减法
 *      
 *      过程:
 *        1. 读取字符串
 *        2. 保证sa>=sb ,a=sa,b=sb
 *        3. 运算 c=a-b
 * */
#include <cstdio>
#include <cstring>

char sa[250],sb[250]; // 存字符串
int a[250],b[250],c[250]; //数据存储

int jianfa(int x[],int y[],int len){
    int i;
    for(i=0;i<len;i++){
        if(x[i] < y[i]) {//借位
            x[i+1]--;
            c[i]=10+x[i]-y[i];
        }
        else{
            c[i]=x[i]-y[i];
        }
        /*
            这样写更快一点
            x[i+1]--;
            c[i] = 10+x[i]-y[i];
            x[i+1] += c[i] /10;
            c[i] = c[i] %10;
        */
    }

    //处理最高位
    int t = len-1;
    while(c[t] == 0 && t >= 0) t--;

    return t+1; //范围
}

int main(){
    memset(a,0,sizeof(a));
    memset(b,0,sizeof(b));
    memset(c,0,sizeof(c));
    int la,lb,lc;
    int i;
    int *pa,*pb;
    /* 读取数据 */
    scanf("%s%s",sa,sb);

    /* 处理数据 */
    la = strlen(sa);
    lb = strlen(sb);

    for(i=0;i<la;i++)
        a[i] = sa[la-1-i] - '0';

    for(i=0;i<lb;i++)
        b[i] = sb[lb-1-i] -'0';

    /* 默认 a - b 我们一定保障 pa>=pb*/
    if(lb >la || (la== lb && strcmp(sa,sb)==-1)){ //满足这些条件 a < b
        printf("-");//flag;
        pa = b;pb = a;
        lc =lb;
    } 
    else {
        pa =a;pb=b;
        lc =la;
    }
    int len = jianfa(pa,pb,lc);

    /* 输出 */
    if(len ==0)
        printf("0");
    else {
        for(i=len-1;i>=0;i--)
            printf("%d",c[i]);
    }

    return 0;
}
\end{lstlisting}


\subsection{亿进制高精度减法}

\begin{lstlisting}
/*============================================================================
* Title : 亿进制高精度减法 
* Author: Rainboy
* Time  : 2016-04-14 11:06
* update: 2016-05-09 10:33
* © Copyright 2016 Rainboy. All Rights Reserved.
*=============================================================================*/

/*  原理模拟减法
 *      
 *      过程:
 *        1. 读取字符串
 *        2. 保证sa>=sb ,a=sa,b=sb
 *        3. 运算 c=a-b
 * */

#include <cstdio>
#include <cstring>

const int base =8;
const int base2 = 100000000;

char sa[250],sb[250]; // 存字符串

int a[250],b[250],c[250]; //数据存储

int jianfa(int x[],int y[],int len){
    int i;
    for(i=0;i<len;i++){
        if(x[i] < y[i]) {//借位
            x[i+1]--;
            c[i]=base2+x[i]-y[i];
        }
        else{
            c[i]=x[i]-y[i];
        }
    }

    //处理最高位
    int t = len-1;
    while(c[t] == 0 && t >= 0) t--;

    return t+1; //范围最后的位数
}

/* 返加 位数 */
/* 用这个 str2num x指向的数组长度比较实际使用的长度长一点*/
int str2num(char x[],int y[]){
    int i,j=0,k=1;
    int lenx= strlen(x);
    for(i=lenx-1;i>=0;i--){
        if(k==base2) j++,k=1;
        y[j] += k*(x[i]-'0');
        k*=10;
    }
    return j+1;
}
int main(){
    memset(a,0,sizeof(a));
    memset(b,0,sizeof(b));
    memset(c,0,sizeof(c));
    int la,lb,lc;
    int i;
    int *pa,*pb;
    /* 读取数据 */
    scanf("%s%s",sa,sb);

    /* 处理数据 */
    str2num(sa,a);
    str2num(sb,b);

    la = strlen(sa);
    lb = strlen(sb);
    /* 默认 a - b 我们一定保障 a>b*/
    if(lb >la || (la== lb && strcmp(sa,sb)==-1)){ //满足这些条件 a < b
        printf("-");//flag;
        pa = b;pb = a;
        lc =lb;
    } 
    else {
        pa =a;pb=b;
        lc =la;
    }

    int len = jianfa(pa,pb,lc);

    /* 输出 */
    if(len == 0)
        printf("0");
    else {
        /*输出最高位*/
        printf("%d",c[len-1]);
        for(i=len-2;i>=0;i--)
            printf("%08d",c[i]);
    }

    return 0;
}
\end{lstlisting}

\subsection{朴素高精度乘法}

\begin{lstlisting}
/*============================================================================
* Title : 朴素高精度乘法
* Author: Rainboy
* Time  : 2016-04-14 11:06
* update: 2016-05-09 10:33
* © Copyright 2016 Rainboy. All Rights Reserved.
*=============================================================================*/
/*  思想:
 *          模拟运算,a[]="123" b[] = "123"
 *          a[1]*b[1]  放在c[1+1]
 *          a[1]*b[0] 放在c[1+0]
 *  原理:
 *      a,b以0下标开始,a[i]*b[i]放在c[i+j]位置上
 * */
#include <cstdio>
#include <cstring>


char s1[250],s2[250]; // 存字符串

int a[250],b[250],c[250]; //数据存储

/* c= a*b 返回位数 */
int mul(int a[],int la,int b[],int lb,int c[]){
    int lc = la+lb;
    int i,j;
    for(i=0;i<la;i++){
        for(j = 0;j<lb;j++){
            c[i+j] =c[i+j] + a[i]*b[j];
        }
    }

    /* 一次处理 */
    for(i=0;i<lc;i++){
        c[i+1] = c[i+1] + c[i] / 10;
        c[i] = c[i] % 10;
    }
    /* 查找最高位 是哪个 */
    /* 最高位的下标只可是 lc-1 或者lc-2 ,如果a,b的最高位不为0 */
    if( c[lc-1] == 0 ) lc--;

    return lc; //返回长度
}

void str2num(char str[],int x[],int lx){
    int i;
    for(i=0;i<lx;i++)
        x[i] = str[lx-1-i] -'0';
}

int main(){
    scanf("%s%s",s1,s2);

    int la,lb,lc;
    int i;

    la = strlen(s1);
    lb = strlen(s2);
    if((la == 1 && s1[0] == '0') || (lb==1 && s2[0] == '0'))
        printf("0");
    else {
        str2num(s1,a,la);
        str2num(s2,b,lb);
        lc = mul(a,la,b,lb,c);
        for(i=lc-1;i>=0;i--)
            printf("%d",c[i]);
    }
    /* 处理最高位 */
    /* n位数 * m位数 ,最多有n+m位,最少n+m-1位 */
    return 0;
}
\end{lstlisting}

\subsection{万进制高精度乘法}

\begin{lstlisting}
/*============================================================================
* Title : 万进制高精度乘法
* Author: Rainboy
* Time  : 2016-04-14 11:06
* update: 2016-06-12 10:11
* © Copyright 2016 Rainboy. All Rights Reserved.
*=============================================================================*/
/*  思想:
 *      四位一存
 * */
#include <cstdio>
#include <cstring>

char sa[250],sb[250]; // 存字符串

int a[250],b[250],c[250]; //数据存储

const int base = 4;
const int base2 =10000;

/* c= a*b 返回位数 */
int mul(int a[],int la,int b[],int lb,int c[]){
    int lc = la+lb;
    int i,j;
    for(i=0;i<la;i++){
        for(j = 0;j<lb;j++){
            c[i+j] =c[i+j] + a[i]*b[j];
        }
    }

    /* 一次处理 */
    for(i=0;i<lc;i++){
        c[i+1] = c[i+1] + c[i] / base2;
        c[i] = c[i] % base2;
    }
    /* 查找最高位 是哪个 */
    /* 最高位的下标只可是 lc-1 或者lc-2 ,如果a,b的最高位不为0 */
    if( c[lc-1] == 0 ) lc--;
    return lc; //返回长度
}

int str2num(char x[],int y[]){
    int i,j=0,k=1;
    int lenx= strlen(x);
    for(i=lenx-1;i>=0;i--){
        if(k==base2) j++,k=1;
        y[j] += k*(x[i]-'0');
        k*=10;
    }
    return j+1;
}

int main(){
    memset(a,0,sizeof(a));
    memset(b,0,sizeof(b));
    memset(c,0,sizeof(c));

    scanf("%s%s",sa,sb);
    int la,lb,lc;
    int i;

    la = strlen(sa);
    lb = strlen(sb);
    if((la == 1 && sa[0] == '0') || (lb==1 && sb[0] == '0'))
        printf("0");
    else {
        la = str2num(sa,a);
        lb = str2num(sb,b);
        /* 处理最高位 */
        /* n位数 * m位数 ,最多有n+m位,最少n+m-1位 */
        lc = mul(a,la,b,lb,c);


        /* 输出最高位 */
        printf("%d",c[lc-1]);
        for(i=lc-2;i>=0;i--)
            printf("%04d",c[i]);
    }
    return 0;
}
\end{lstlisting}

