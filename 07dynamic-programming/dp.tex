\section{动态规化}

\subsection{相关概念}

一句话概念:\textbf{就是子问题状态的迁移过程}

解法步骤:

\begin{itemize}
 \item 确定问题是DP
 \item 想出所有的DP状态
 \item 想出状态是如果迁移的:状态转移方程
 \begin{itemize}
 \item 状态的前趋很好找到--->递推的DP
 \item 状态的前趋不好找到--->记忆化搜索
 \end{itemize}
\end{itemize}

\subsection{线性DP}

线型动态规划问题,最典型的特征就是状态都在一条线上,并且位置固定,问题一般都规定只能从前往后取状态,解决的办法是根据前面的状态特征,选取最优状态作为决策进行转移。 
\begin{itemize}
  \item 设前 i个点的最优值,研究前 i-1 个点与前 i个点的最优值, 
  \item 利用第 i个点决策转移 
\end{itemize}

状态转移方程一般可写成:

$$
f_i(k) = min\left \{ f_{i-1}\ or\ j( k') + u(i,j)\ or\ u(i,i-1) \right \} 
$$

\subsubsection{最长不下降子序列LIS-$n^2$}

\begin{lstlisting}
int a[100];
int f[100];
int LIS(int n){
    int i,j,ans=0;
    for (i=1;i<=n;i\){
        scanf("%d",&a[i]);
        f[i]=1;
    }
    
    for (i=2;i<=n;i++){
        for (j=1;j<=i-1;j++){
            if(a[i] > a[j])
                f[j] = max(f[i],f[j]+1);
        }
    }

    for (i=1;i<=n;i++){
        if(f[i] > ans)
            ans = f[x];
    }
    return ans;
}
\end{lstlisting}


\textbf{二分查找法} $n \times logn$代码由Riolu(2014)提供

\begin{lstlisting}
int a[n],c[n],f[n],m;
//a 为数组,f存最大长度
//c是符合条件的最长非降序列
//c[i]表示长度为i的LIS最小右端值

memset(c,0x7f,siezof);

int LIS(){
    int i;
    for(i=1;i<=n;i++){
        int min=1,max=i;

        while(min<max){     //寻找最大c[x],使c[x]<=a[i]
            int mid  = (min+max)>>1;
            if(c[mid] <= a[i])
                min = mid+1;
            else
                max= mid;
        }

        f[i]=min;   //保有答案
        c[min]=a[i];//更新最小值
        return f[n];
    }
}
\end{lstlisting}

\subsubsection{最长公共子序列}

\textbf{状态转移方程:}

$$
f(i,j) = 
\left\{\begin{matrix}
f(i-1,j)\\ 
f(i,j-1)\\ 
f(i-1,j-1)+a_i==b_j\ ?\ 1:0
\end{matrix}\right.
$$

\begin{lstlisting}
int f[1001][1001];

int LCS(int *a,int m,int *b,int n){ // m,n分别是a,b中元素的个数
    int i,j;
    memset(f,0,sizeof(f));
    for (i=1;i<=m;i++)
    for (j=1;j<=n;j++){
        if (a[i] == b[j]) f[i][j] = f[i-1][j-1]+1;
        f[i][j] = max(f[i][j],max(f[i-1][j],f[i][j-1]));
    }
}
\end{lstlisting}


\subsection{区域型DP}

\begin{itemize}
    \item \textbf{特征:}能将问题分解成为两两合并的形式 
    \item \textbf{合并:}意思就是将两个或多个部分进行整合,当然也可以反过来,也就是是将一个问题进行分解成两个或多个部分.
    \item \textbf{求解:}对整个问题设最优值,\textbf{枚举合并点},将问题分解成为左右两个部分,最后将左右两个部分的最优值进行合并得到原问题的最优值。有点类似分治算法的解题思想。 
    \item \textbf{典型试题:}整数划分,凸多边形划分、石子合并、多边形合并、能量项链等、数字游戏。
    \item \textbf{特征:}能将问题分解成为两两合并的形式 
\end{itemize}

\subsubsection{石子合并}
