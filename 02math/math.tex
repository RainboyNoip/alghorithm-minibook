\section{数学相关}

\subsection{GCD:欧几里得算法}

\subsubsection{递归实现}
\begin{lstlisting}
int gcd(int a,int b){
    if (b==0) return a;
    return gcd(b, a%b);
}
\end{lstlisting}

\subsubsection{非递归实现}
\begin{lstlisting}
int gcd(int a,int b){
    int t;
    while(b != 0){
        t=a;a=b;b= t % b;
    }
    return a;
}
\end{lstlisting}

\subsection{EX-GCG:拓展欧几里得算法}
\begin{lstlisting}
int ex_gcd(int a,int b,int c,int &x,int &y){
    if(b == 0){
        x=c/a;
        y=0;
        return a;
    }
    int ans = ex_gcd(b,a%b,c2,x,y);
    int tmp =x;
    x = y;
    y = tmp -a/b*y;
    return ans;
}
\end{lstlisting}

\subsection{LCM:最小公倍数}
\textbf{原理:}
$$a \times b = lcm(a,b) \times gcd(a,b)$$
\begin{lstlisting}
int lcm(int a,int b){
    /* 原理 a*b = lcm(a,b)*gcd(a,b) */
    return (a / gcd(a,b) *b); //注意先 除
}
\end{lstlisting}

\subsection{筛法求素数}
\begin{lstlisting}
#include <cstdio>
#include <cstring>

bool f[100000010];
int su[100010];
int sj,n=0,i,j;

void ShaSuShu(){
    memset(f,1,sizeof(f));
    for (i=2;i<=sj;i++){ //筛选
        if(f[i]){
            for(j=i<<1;j<=sj;j+=i)
                f[j] =false;
        }
    }
    for (i=2;i<=sj;i++){
        if( f[i]){
            n++;
            su[n]=i; //保存
        }
    }
}

int main(){
    scanf("%d",&sj);
    ShaSuShu();
    int i;
    for(i=1;i<=n;i++)
        printf("%d ",su[i]);
    return 0;
}
\end{lstlisting}

\subsection{常用递推公式}

\textbf{加法原理}:

做一件事情,完成它可以有$n$类办法,在第一类办法中有$m_1$种不同的方法,在第二类办法中有$m_2$种不同的方法,……,在第$n$类办法中有mn种不同的方法。那么完成这件事共有$N=m1+m2+…+mn$种不同的方法。

\textbf{乘法原理}:

做一件事情,完成它需要分成n个步骤,做第一步有$m_1$种不同的方法,做第二步有$m_2$种不同的方法,……,做第n步有种$m_n$不同的方法,那么完成这件事有$N+m_1 \times m_2 \times … \times m_n$种不同的方法。

\textbf{两个原理的区别}:

一个与分类有关,一个与分步有关;加法原理是\textbf{分类完成},乘法原理是\textbf{分步完成}。

\subsection{排列组合}

\subsubsection{排列}

\textbf{概念:}


从$n$个不同元素中,任取$m\times(m \leqslant n)$个元素按照一定的顺序排成一列,叫做从$n$个不同元素中取出$m$个元素的一个排列。

\textbf{排列数:}
从$n$个不同元素中取出$m(m\leqslant n)$个元素的所有排列的个数,叫做从$n$个不同元素中取出$m$个元素的排列数,用符号
\textbf{计算公式:}

$$A_n^m=n(n-1)(n-2)......(n-m+1)=\frac{n!}{(n-m)!}$$


\subsubsection{组合}

\textbf{概念:}

从n个不同元素中,任取$m(m\leqslant n)$个元素并成一组,叫做从$n$个不同元素中取出$m$个元素的一个组合。

 \textbf{组合数:}
 从$n$个不同元素中取出$m(m\leqslant n)$个元素的所有组合的个数,叫做从$n$个不同元素中取出$m$个元素的组合数,用符号$C_n^m$表示

 \textbf{计算公式:}

 $$C_n^m =\frac{ A_n^m}{m!}=\frac{n(n-1)(n-2)......(n-m+1)}{m!}=\frac{n!}{(n-m)!}$$

\textbf{组合恒等式:}


\begin{align}
&C_n^m=C_n^{n-m}  \label{eq:rel1} \\
&C_n^m=C_{n-1}^{m-1}+C_{n-1}^{m}  \label{eq:rel2} \\
&C_n^n+C_{n+1}^n+C_{n+2}^n+...+C_{n+r}^n=C_{n+r+1}^{n+1} \label{eq:rel3} \\
&C_n^0+C_n^1+C_n^2+...+C_n^n=2^n \label{eq:rel4} \\
&C_n^0+C_n^2+C_n^4+...=C_n^1+C_n^3+C_n^5+...=2^{n-1} \label{eq:rel5} \\
\end{align}


\subsection{高中数学知识补充}

 n个人围着一张圆桌坐在一起,共有(n-1)!种坐法。

 把r个相同的球放到n个不同颜色的盒子中去,共有$C_{n+r-1}^r$种方法

 从n个排成一排的数中取m个数,且数字之间互不相邻,共有$C_{n-m+1}^r$种取法

\subsection{二项式定理}

\textbf{公式:}

$$(a+b)^n=C_n^0a^nb^0+C_n^1a^{n-1}b^1+......+C_n^ra^{n-r}b^r+......+C_n^na^{0}b^n$$

其中$a^{n-r}b^r$的二项式系数为$C_n^r$

\subsection{catalan 数列}

\textbf{Catalan数列:}

$$ 1,2,5,14,42,132,429……, $$

\textbf{通项公式:}

$$ h(n) = C_{2n}^{n} /(n+1)$$


\textbf{应用:}

01串, 出栈序列

\subsection{鸽巢原理(抽屉原理)}

 简单形式:如果$n+1$个物体被放进$n$个盒子,那么至少有一个盒子包含两个或更多的物体。

加强形式:令$q_1,q_2,...,q_n$为正整数。如果将$q_1+q2+...+qn-n+1$个物体放入$n$个盒子内,那么或者第一个盒子至少含有$q_1$个物体,或者第二个盒子至少含有$q_2$个物体,…,或者第n个盒子含有$q_n$个物体

\textbf{推论1:}

$m$只鸽子进$n$个巢,至少有一个巢里有$m/n$只鸽子

\textbf{推论2:}

$n(m-1)+1$只鸽子进n个巢,至少有一个巢内至少有m只鸽子。
  - 推论3:若$m_1,m_2,...,m_n$是正整数,且$\frac{m_1+...+m_n}{n}>r-1$.则至少有一个不小于$r$。


\subsection{容斥原理}

公式复杂,随后给出

\subsection{进制转换}

\textbf{N进制转十进制(秦九韶算法)}

\textbf{bit[]} 对应的N进制位

\textbf{top}是\textbf{bit[]}最高位的下标,下标从0开始

\begin{lstlisting}
int convertFrom(int base,int *bit,int top){
    int ans = 0;
    for(int i =top;i>=0;i--){
        ans*=base;
        ans+=bit[i];
    }
    return ans;
}
\end{lstlisting}

\textbf{十进制转K进制(短除法)}

\textbf{bit[]} 对应的N进制位

\textbf{top}是\textbf{bit[]}最高位的下标,下标从0开始

\begin{lstlisting}
void convertTo(int num,int base,int *bit,int &top){
    top=-1;
    do{
        bit[++top] = num % base;//可以直接输出 num % base
        num/=base;
    }while(num>0);
}
\end{lstlisting}

\subsection{组合相关算法}

\subsubsection{加法递推-$O(n^2)$}

\textbf{使用的公式:}

$$C_{n}^{m} = C_{n-1}^{m-1}+C_{n-1}^{m}$$

\textbf{边界:}
$$
C_n^0=C_n^n = 1
$$

这个也是二项式定理,杨辉三角

\begin{lstlisting}
int c[1001][1001]; //根据实际需要开数组,必要的时候彩高精度,
                   //具体看 高精度组合
............
memset(c,0,sizeof(c));
int i,j;
c[0][0] =1; //边界
for(i=1;i<=n;i++){
    c[i][0] = 1;
    for(j=1;j<=i;j++){
        c[i][j] = c[i-1][j-1]+c[i-1][j];
    }
}
\end{lstlisting}

\subsubsection{乘法递推-$O(n)$}

\begin{lstlisting}
int c[1001]; //必要的时候使用组合高精
.............
c[0]=1;
if( m >n-m) m = n-m;
for(int i =1;i<=m;i++){
    c[i] = (n-i+1)*c[i-1] / i;
}
\end{lstlisting}

\subsubsection{一般组合的产生}

从$n$个元素里取$m$个元素,有多少种取法,把所有的取法都输出

一个合法的组合有这样的特点

\textbf{排在后面的数字一定严格大于左面的数字}

\begin{lstlisting}
#include <cstdio>

int n,m;

int stack[10000];
void print(){
    int i;
    for (i=1;i<=m;i++)
        printf("%d ",stack[i]);
    printf("\n");
}
void dfs(int depth,int p){
    if(depth == m+1){
        print();
        return ;
    }
    int i;
    /*  p+1 <= k <=n-(m-depth) */
    for (i=p+1;i<=n-(m-depth);i++){
        /* 由于后面的数一定比较前面的数大,不用标记 */
        stack[depth]=i;
        dfs(depth+1,i);
    }

}

int main(){
    scanf("%d%d",&n,&m);
    dfs(1,0);
    return 0;
}
\end{lstlisting}



\subsubsection{全组合}

有$n$个元素,求这$n$个数构成的集合的所有非空子集

\begin{lstlisting}
#include <cstdio>
int n;
int item[10001]={0};

void full_combinatiom(int pos,int val){
	int i;
	/* i < pos pos代表正要选的位置,还没有选*/
    for(i=1;i<pos;i++)//输出已有的值
        printf("%d ",item[i]);//进入函数就输出
    printf("\n");
	
    for(i=val;i<=n;i++){
        item[pos] = i;
		/*位置+1,当前可以选的最小值+1*/
        full_combinatiom(pos+1,i+1);
		
    }
}

int main(){
	
	n = 3;
	full_combinatiom(1,1);
	return 0;
}
\end{lstlisting}



可以用一个整数来表示一个集合或者组合,因为第一位不是0就是1:具体可以看容斥原理

\subsubsection{由上一组合产生下一组合}

从右边向左寻找可以往下取的元素的数,位置为j
数列的j位到n位的重取元素
右边的元素一定严格大于左侧的元素

代码有可能错误,但是思想正确,因为没有验证

\begin{lstlisting}
int a[M];//a[i] 字典序的最小排序
....

int j =m-1;
while((j >=0) && (a[j]==n-(m-1-j))) j--;
if( i >=0){
    a[j]++;
    int k;
    for(k=j+1;k<m;k++) 
        a[k]=a[k-1]+1;
}
\end{lstlisting}



\subsection{排列相关算法}

\subsubsection{排列数的计算}

根据数学公式:

$$
n!=n \times (n-1) \times (n-2) \times ......\times 1
$$

\begin{lstlisting}
typedef long long LL;
LL n,ans=1;
scanf("%d",&ans);
while(n !=0){
    ans*=n;
    n-=1;
};
\end{lstlisting}



\subsubsection{全排列的产生}

对数字\textbf{1-->n}进行排序,输出每一种排序方法

当然是用DFS,对每个位置进行试探.

\textbf{DFS原则:}

\textbf{在某个位置下,选一个没有被选过的数,边界是n}

\begin{lstlisting}
#include <cstdio>
int n;
/* 0代表没有被使用过 */
bool vis[10000]={0}; 
/* 存数据 */
int stack[10000];

void print(){
    int i;
    for (i=1;i<=n;i++)
        printf("%d ",stack[i]);
    printf("\n");
}

void dfs(int depth){
    if( depth == n+1){
        print();
        return ;
    }
    int i;
    for (i=1;i<=n;i++){
        if( vis[i] == 0){//如果这个数没有用过,就用
            vis[i] = 1;
            stack[depth]=i;
			dfs(depth+1);
            vis[i] = 0; //设为没有用过,还可以给其它位置用
        }
    }
}

int main(){
    scanf("%d",&n);
    dfs(1);
	return 0;
}
\end{lstlisting}

\subsubsection{由上一个排列产生下一个排列}

1. 从右往左寻找第一个小于右边的数,位置j.
2. 在是位置上的右边寻找大于$a_j$r的最小数字$a_k$
3. 将$a_j$与$a_k$的值进行交换
4. 将数列的$j+1$位到$n$倒转

代码没有验证

\begin{lstlisting}
int a[N];
int j,k,p,q,temp;

j=(n-1)-1;
while(( j<=0) && (a[j]>a[j+1])) j--;

if(j >=0){
    k=n-1;
    while(a[k] < a[j]) k--;
    swap(a[j],a[k]);
    for(p=j+1,q=n-1;p<q;p++,q--){
        swap(a[p],a[q]);
    }
}
\end{lstlisting}
