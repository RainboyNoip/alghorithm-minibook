\section{排序算法}

\subsection{冒泡排序}

\textbf{\underline{一句话算法:}} 数组下标从1开始,第一趟循环len-1次,第二趟循环len-2次,第i趟循环len-i次 

\begin{lstlisting}
int a[10]= {0,3,2,4,9,1,5,7,6,8};
for(i=1;i<=n-1;i++)
    for(j=1;j<=n-i;j++)
       if(a[j] > a[j+1])   
        tmp =a[j],a[j] =a[j+1],a[j+1]=tmp;
\end{lstlisting}

\subsection{归并排序}

归并排序可以求逆序对:\textbf{火柴棒排序}
\textbf{归并排序的交换次数就是这个数组的逆序对个数}

\begin{lstlisting}
int a[] ={1,7,3,6,5,2};
int tmp[100]; //临时存储的中间数组

void merge_sort(int s,int t){
    int mid,i,j,k;
    if(s==t) return ; //如果区间只有一个数,就返回
    mid = (s+t)>>1; //取中间的点
    merge_sort(s,mid);
    merge_sort(mid+1,t);

    i=s;
    j=mid+1;
    k=s;

    while(i<=mid && j<=t){
        if( a[i] <=a[j]){
            tmp[k]=a[i];k++;i++;
        } else {
            tmp[k]=a[j];j++;k++;
        }
    }

    while(i<=mid) { tmp[k]=a[i];k++;i++;};
    while(j<=t)   { tmp[k]=a[j];k++;j++;};

    for(i=s;i<=t;i++)
        a[i]=tmp[i];
}
\end{lstlisting}



\subsection{★快速排序}

\begin{lstlisting}
int a[]={6,2,7,3,8,9};
void quicksort(int l,int r){
    if( l <=r ){
    int s=l,t=r;
    int key =a[l]; // 取第一个值为key
    while(s < t){
        while( s <t && a[t] >= key) --t;// 如果a[t] >= key,t下标不停变小,直到a[t] < key
        if(s < t) a[s++] = a[t];        //停下来的时候,看一看,是不是到中点,如果不是 交换值
        while(s<t && a[s] <= key) ++s;  //如果a[s] <= key  s的下标不停变大,直到a[s] > key
        if(s<t ) a[t--] = a[s];         //停下来的时候,看一看,是不是到了中点,如果不是,交换值
    }
    a[s] = key;                         //上面while停止的时候,一定是s ==t
    quicksort(l,s-1);
    quicksort(s+1,r);
}
}
\end{lstlisting}