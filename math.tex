\section{数学相关}

\subsection{常用递推公式}

\textbf{加法原理}:

做一件事情,完成它可以有$n$类办法,在第一类办法中有$m_1$种不同的方法,在第二类办法中有$m_2$种不同的方法,……,在第$n$类办法中有mn种不同的方法。那么完成这件事共有$N=m1+m2+…+mn$种不同的方法。

\textbf{乘法原理}:

做一件事情,完成它需要分成n个步骤,做第一步有$m_1$种不同的方法,做第二步有$m_2$种不同的方法,……,做第n步有种$m_n$不同的方法,那么完成这件事有$N+m_1 \times m_2 \times … \times m_n$种不同的方法。

\textbf{两个原理的区别}:

一个与分类有关,一个与分步有关;加法原理是\textbf{分类完成},乘法原理是\textbf{分步完成}。

\subsection{排列组合}

\subsubsection{排列}

\textbf{概念:}


从$n$个不同元素中,任取$m\times(m \leqslant n)$个元素按照一定的顺序排成一列,叫做从$n$个不同元素中取出$m$个元素的一个排列。

\textbf{排列数:}
从$n$个不同元素中取出$m(m\leqslant n)$个元素的所有排列的个数,叫做从$n$个不同元素中取出$m$个元素的排列数,用符号
\textbf{计算公式:}

$$A_n^m=n(n-1)(n-2)......(n-m+1)=\frac{n!}{(n-m)!}$$


\subsubsection{组合}

\textbf{概念:}

从n个不同元素中,任取$m(m\leqslant n)$个元素并成一组,叫做从$n$个不同元素中取出$m$个元素的一个组合。

 \textbf{组合数:}
 从$n$个不同元素中取出$m(m\leqslant n)$个元素的所有组合的个数,叫做从$n$个不同元素中取出$m$个元素的组合数,用符号$C_n^m$表示

 \textbf{计算公式:}

 $$C_n^m =\frac{ A_n^m}{m!}=\frac{n(n-1)(n-2)......(n-m+1)}{m!}=\frac{n!}{(n-m)!}$$

\textbf{组合恒等式:}


\begin{align}
&C_n^m=C_n^{n-m}  \label{eq:rel1} \\
&C_n^m=C_{n-1}^{m-1}+C_{n-1}^{m}  \label{eq:rel2} \\
&C_n^n+C_{n+1}^n+C_{n+2}^n+...+C_{n+r}^n=C_{n+r+1}^{n+1} \label{eq:rel3} \\
&C_n^0+C_n^1+C_n^2+...+C_n^n=2^n \label{eq:rel4} \\
&C_n^0+C_n^2+C_n^4+...=C_n^1+C_n^3+C_n^5+...=2^{n-1} \label{eq:rel5} \\
\end{align}


\subsection{高中数学知识补充}

 n个人围着一张圆桌坐在一起,共有(n-1)!种坐法。

 把r个相同的球放到n个不同颜色的盒子中去,共有$C_{n+r-1}^r$种方法

 从n个排成一排的数中取m个数,且数字之间互不相邻,共有$C_{n-m+1}^r$种取法

\subsection{二项式定理}

\textbf{公式:}

$$(a+b)^n=C_n^0a^nb^0+C_n^1a^{n-1}b^1+......+C_n^ra^{n-r}b^r+......+C_n^na^{0}b^n$$

其中$a^{n-r}b^r$的二项式系数为C_n^r$

\subsection{catalan 数列}

\textbf{Catalan数列:}

$$ 1,2,5,14,42,132,429……, $$

\textbf{通项公式:}

$$ h(n) = C_{2n}^{n} /(n+1)$$


\textbf{应用:}

01串, 出栈序列

\subsection{鸽巢原理(抽屉原理)}

 简单形式:如果$n+1$个物体被放进$n$个盒子,那么至少有一个盒子包含两个或更多的物体。

加强形式:令$q_1,q_2,...,q_n$为正整数。如果将$q_1+q2+...+qn-n+1$个物体放入$n$个盒子内,那么或者第一个盒子至少含有$q_1$个物体,或者第二个盒子至少含有$q_2$个物体,…,或者第n个盒子含有$q_n$个物体

\textbf{推论1:}

$m$只鸽子进$n$个巢,至少有一个巢里有$m/n$只鸽子

\textbf{推论2:}

$n(m-1)+1$只鸽子进n个巢,至少有一个巢内至少有m只鸽子。
  - 推论3:若$m_1,m_2,...,m_n$是正整数,且$\frac{m_1+...+m_n}{n}>r-1$.则至少有一个不小于$r$。


\subsection{容斥原理}

公式复杂,随后给出

\subsection{进制转换}

\textbf{N进制转十进制(秦九韶算法)}

\textbf{bit[]} 对应的N进制位

\textbf{top}是\textbf{bit[]}最高位的下标,下标从0开始

\begin{lstlisting}
int convertFrom(int base,int *bit,int top){
    int ans = 0;
    for(int i =top;i>=0;i--){
        ans*=base;
        ans+=bit[i];
    }
    return ans;
}
\end{lstlisting}

\textbf{十进制转K进制(短除法)}

\textbf{bit[]} 对应的N进制位

\textbf{top}是\textbf{bit[]}最高位的下标,下标从0开始

\begin{lstlisting}
void convertTo(int num,int base,int *bit,int &top){
    top=-1;
    do{
        bit[++top] = num % base;//可以直接输出 num % base
        num/=base;
    }while(num>0);
}
\end{lstlisting}
