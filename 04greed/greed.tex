\section{贪心算法}

\subsection{贪心策略的基本思想}

\textbf{定义}

贪心法是一种解决最优问题的策略。它是从问题的初始解出发,按照当前最佳的选择,把问题归纳为更小的相似的子问题,并使子问题最优,再由子问题来推导出全局最优解。


贪心是一种解题策略,也是一种解题思想


使用贪心方法需要注意局部最优与全局最优的关系,选择当前状态的局部最优并不一定能推导出问题的全局最优。


\subsection{贪心问题解题过程}

(1).怎么知道问题是否能用贪心策略求解

\textbf{贪心选择特性:}即通过局部的贪心选择来达到问题的全局最优解。

\textbf{最优子结构性质:}即原问题的最优解包含子问题的最优解。

(2).如何选择贪心标准,以得到问题的最优解